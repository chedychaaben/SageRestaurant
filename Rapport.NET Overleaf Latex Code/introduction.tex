\chapter*{Introduction générale}
\addcontentsline{toc}{chapter}{Introduction générale} 

\markboth{Introduction générale}{}

Dans un monde de plus en plus connecté et numérique, la gestion efficace des systèmes d'information est essentielle pour garantir la bonne marche des entreprises, y compris celles du secteur de la restauration. L'automatisation des processus, la sécurisation des données et l'amélioration de l'expérience utilisateur deviennent des éléments clés pour répondre aux défis contemporains. Dans ce contexte, la gestion d'un restaurant ne se limite plus à des tâches manuelles, mais s'appuie sur des solutions logicielles intelligentes pour optimiser les opérations quotidiennes, comme la gestion des commandes et du stock des produits.

Ce projet de développement d'une application web pour un restaurant se focalise sur l'utilisation d'une API, offrant ainsi une meilleure flexibilité et sécurité, plutôt que sur un site web traditionnel. 

L'objectif est de permettre une gestion fluide des commandesau sein du système tout en garantissant un accès sécurisé aux différentes fonctionnalités de l'application. 

Afin d’illustrer et de structurer efficacement cette application, des diagrammes de classe et des cas d’utilisation ont été élaborés pour fournir une vue d'ensemble des entités et de leurs interactions dans l’application. En outre, une série d'endpoints API ont été développés pour permettre la communication entre le serveur et le client, facilitant ainsi les opérations de gestion des commandes.
