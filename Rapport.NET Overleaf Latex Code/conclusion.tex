
\section*{Conclusion}

En conclusion, l'application de gestion des commandes de restaurant que nous avons développée représente une solution complète et moderne pour la gestion des commandes, de l'authentification des utilisateurs et a la consultation du menu. Cette application a été conçue pour répondre à un besoin spécifique dans le domaine de la restauration, en facilitant la prise de commande en ligne et la gestion des données utilisateurs dans un environnement sécurisé.

L'implémentation de l'authentification basée sur JWT (JSON Web Token) assure une gestion fluide et sécurisée des sessions utilisateurs, tout en garantissant que seules les requêtes authentifiées peuvent accéder aux ressources protégées de l'application. Le processus d'authentification est non seulement sécurisé, mais aussi évolutif, permettant à l'application de supporter une base d'utilisateurs croissante sans nécessiter de stockage des sessions côté serveur.

Le Repository Pattern a été appliqué pour maintenir une séparation claire entre la logique métier et la gestion des données. Cela permet une meilleure organisation du code, une facilité de maintenance et une meilleure testabilité des différentes parties de l'application. L'utilisation de ce pattern a permis de centraliser l'accès aux données et de simplifier les opérations CRUD pour les entités principales.

Les différents endpoints de l'API ont été soigneusement développés pour répondre aux besoins des utilisateurs, tout en offrant une interface claire et intuitive pour l'administration et les clients. La sécurisation des endpoints par JWT et la gestion des rôles d'utilisateur (comme l'administrateur et les client) assurent une utilisation sécurisée et flexible de l'application.

De plus, les diagrammes de classe et de cas d'utilisation ont permis de clarifier la structure de l'application, illustrant les relations entre les entités et les principales interactions entre les acteurs du système. Ces diagrammes ont facilité la compréhension du fonctionnement interne de l'application et ont guidé le développement des différentes fonctionnalités.

En somme, cette application démontre l'importance d'une architecture solide et d'une gestion sécurisée des données dans les applications modernes, en particulier celles qui manipulent des informations sensibles. Avec une bonne base en place, il est possible d'ajouter de nouvelles fonctionnalités, comme des systèmes de notifications ou des modules de recommandation pour améliorer l'expérience client. L'utilisation de technologies robustes et la mise en œuvre des meilleures pratiques garantissent la pérennité et la scalabilité de l'application à long terme.
